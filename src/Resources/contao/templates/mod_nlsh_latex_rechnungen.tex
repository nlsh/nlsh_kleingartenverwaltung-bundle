% Standardvorlage für Rechnungen des Kleingartenvereins "Bullenwiese" Prenzlau e.V.
%
% Erläuterungen beziehen sich auf dieses PDF- Dokument
% http://sunsite.informatik.rwth-aachen.de/ftp/pub/mirror/ctan/macros/latex/contrib/koma-script/doc/scrguide.pdf


\documentclass%%
%---------------------------------------------------------------------------
  [fontsize = 12pt,                % Schriftgroesse
%-----------------------------------------------------------------
% Optionen der Dokumentenklasse
% Satzspiegel
   paper            = a4,         % Papierformat
   enlargefirstpage = on,         % Erste Seite anders
   pagenumber       = headright,  % Seitenzahl unten Rechts; S. 20 
   firstfoot        = on,         % erster Fuß ja/nein
%-----------------------------------------------------------------
% Layout
   headsepline = on,              % Linie unter der Seitenzahl
   parskip     = half,            % Abstand zwischen Absaetzen
%-----------------------------------------------------------------
% Briefkopf und Anschrift
   firsthead     = on,            % Briefkopf an/aus; S 179
   fromalign     = left,          % Plazierung des Briefkopfs; S. 180
   fromrule      = off,           % Linie im Absender (aftername, afteraddress); S. 181
   fromphone     = on,            % Absender- Telefonnummer anzeigen; S. 185
   fromfax       = off,           % Absender- Faxnummer anzeigen; S. 185
   fromemail     = off,           % Absender- Emailadresse anzeigen; S. 185
   fromurl       = off,           % Absender- Homepage anzeigen; S. 185
   symbolicnames = on,            % Symble oder Text o.g. Absenderangaben; S. 185
   fromlogo      = off,           % Firmenlogo anzeigen; S. 188
   addrfield     = on,            % Adressfeld fuer Fensterkuverts; S. 192
   backaddress   = on,            % ...und Absender im Fenster; S. 193
   subject       = beforeopening, % Plazierung der Betreffzeile; S. 202
   locfield      = narrow,        % zusaetzliches Feld fuer Absender
   						          % abhängig von fromalign; S. 195
   foldmarks     = on,            % Faltmarken setzen off/on, oder hier DIN; Seite 177
   numericaldate = off,           % Datum numerisch ausgeben; S. 196
   							      % off: xx. string xxx
   							      % on : xx. xx. xxxx
   refline       = narrow,        % Geschaeftszeile im Satzspiegel; S. 198
   	                              % wenn keine Variablen gesetzt, dann nicht gezeigt
%-----------------------------------------------------------------
% Formatierung
   draft         = off            % Entwurfsmodus, als Formatierungshilfe -> on/off
]{scrlttr2}

%---------------------------------------------------------------------------
% zu ladene Packete
\usepackage[utf8]{inputenc}      % ein Muss für Deutsche Texte
\usepackage[T1]{fontenc}         % ein Muss für Deutsche Texte
\usepackage[ngerman]{babel}      % ein Muss für Deutsche Texte
\usepackage{lmodern}             % ein Muss für Deutsche Texte

% zusätzliche Pakete
\usepackage{ulem}                % durchstreichen von Text
\usepackage{lipsum}              % zum Test Text simulieren
\usepackage{color}               % Farbe hinzufügen

% Briefkörper bündig am Briefkopf ausrichten

\setlength{\oddsidemargin}{\useplength{toaddrhpos}}
\addtolength{\oddsidemargin}{-1in}
\setlength{\textwidth}{\useplength{firstheadwidth}}

%---------------------------------------------------------------------------
% Variablen; S. 185

%Absender
\setkomavar{fromname}{Kleingartenverein „Bullenwiese“ Prenzlau e.V.}
\setkomavar{fromaddress}{Seeweg 3\\17291 Prenzlau}
\setkomavar{fromphone}{+49 172 951 590}
\setkomavar{fromemail}{nils@nilsheinold.de}
\setkomavar{frombank}{Volksbank Uckermark e.V.\\IBAN DE22 1509 1704 0300 0039 58}

%Layout

% Kopfzeile --------------------------------------------------------
\newcommand{\kopfZeile}{
	\makebox{\LARGE\textbf{\usekomavar{fromname}}}
	\makebox{
		\rule[10mm]{13cm}{0mm}\scriptsize
		\parbox[t]{\textwidth}{
			\usekomavar*{fromphone}\usekomavar{fromphone} \\
			\usekomavar*{fromemail}\usekomavar{fromemail} \\ \\
			\begin{tabular}[t]{l@{}}%
				\multicolumn{1}{@{}l@{}}{%
				\usekomavar*{frombank}:}\\
				\usekomavar{frombank}
			\end{tabular}%
		}
	}
}

% Fußzeile --------------------------------------------------------
\newcommand{\fussZeile}{
	\parbox[t]{\textwidth}{\footnotesize
		\begin{tabular}[t]{l@{}}%
			\multicolumn{1}{@{}l@{}}{Vereinsvorsitzender:}\\
			Wilfried Launhardt\\
		\end{tabular}%
		\hfill
		\begin{tabular}[t]{l@{}}%
			\multicolumn{1}{@{}l@{}}{Finanzamt:}\\
			Angermünde : St- Nr.: 062/141/01929
		\end{tabular}%
		\hfill
		\begin{tabular}[t]{l@{}}%
			\multicolumn{1}{@{}l@{}}{Vereinsregister:}\\
			Amtsgericht Neuruppin VR 2520 NP
		\end{tabular}%
	}%
}	
% Kopf erste Seite --------------------------------------------------------

\setkomavar{firsthead}{\kopfZeile}

% Fuß erste Seite ---------------------------------------------------------

\setkomavar{firstfoot}{\fussZeile}

% Fuß Folgeseiten ---------------------------------------------------

%---------------------------------------------------------------------------
% Beginn des Briefes
\begin{document}

%---------------------------------------------------------------------------
% Global

\setkomavar{place}{Prenzlau} % den Ort vor das Datum des Briefes stellen
%\setkomavar{date}{1.1.1999} % Datum des Briefes, wenn nicht gesetzt, aktuelles
						     % wenn gesetzt, dann numericaldate außer Kraft
						     % S. 196

%---------------------------------------------------------------------------
% Briefinterna

% Geschäftszeile S. 197
%\setkomavar{yourref}{Ihr Zeichen}         % Ihr Zeichen
%\setkomavar{yourmail}{ihr schreiben von}  % Ihr Schreiben vom
%\setkomavar{myref}{mein zeichen}          % Unser Zeichen
%\setkomavar{customer}{kundennummer}       % Kundennummer
%\setkomavar{invoice}{rgnummer}            % Rechnungsnummer
%\setkomavar{date}{date}                   % Datum

% Signatur
\setkomavar{signature}{- Wilfried Launhardt -\\Vereinsvorsitzender}
%\renewcommand*{\raggedsignature}{\raggedright}  % Rechtsausrichtung der Signatur; S. 211           
%---------------------------------------------------------------------------

	\begin{letter}{Name, Vorname\\ Straße Hausnummer\\ PZL Ort}

		\setkomavar{title}{Rechnung Nr 2015/....}        % Brieftitel, S. 200/ 201 
					
		\opening{Anrede}
		
		Für das Geschäftsjahr 2015 berechnen wir Ihnen die von uns für Sie verauslagten
		Beträge, entsprechend der Ablesung der Messgeräte über Ihren Verbrauch.

		Es sind die Mitgliedsbeiträge für das Jahr 2016 und die Pacht für das Jahr 2016 fällig.

		Im Einzelnen bitten wir Sie, folgende Beträge zu entrichten: 
		
		\addvspace{1cm}
		\addmargin{1cm}
		\begin{tabular}{p{10cm}l}
		• & • \\ 
		• & • \\ 
		• & • \\ 
		• & • \\ 
		• & • \\ 
		\hline 
		\end{tabular} 
		\addmargin{-1cm}
		\addvspace{1cm}

		Sollten von Ihnen Abbuchungsaufträge für Mitgliedsbeitrags- und Pachtzahlungen erteilt
		worden sein, bitten wir diese entsprechend zu berücksichtigen. 
		
		\closing{Mit freundlichen Grüßen}
		
		%\ps PS: HDL % wenn gesetzt, dann PS- Zeile; S. 166

	\end{letter}
	
		\begin{letter}{Name, Vorname\\ Straße Hausnummer\\ PZL Ort}

		\setkomavar{title}{Rechnung Nr 2015/....}        % Brieftitel, S. 200/ 201 
					
		\opening{Anrede}
		
		Für das Geschäftsjahr 2015 berechnen wir Ihnen die von uns für Sie verauslagten
		Beträge, entsprechend der Ablesung der Messgeräte über Ihren Verbrauch.

		Es sind die Mitgliedsbeiträge für das Jahr 2016 und die Pacht für das Jahr 2016 fällig.

		Im Einzelnen bitten wir Sie, folgende Beträge zu entrichten: 
		
		\addvspace{1cm}
		\addmargin{1cm}
		\begin{tabular}{p{10cm}l}
		• & • \\ 
		• & • \\ 
		• & • \\ 
		• & • \\ 
		• & • \\ 
		\hline 
		\end{tabular} 
		\addmargin{-1cm}
		\addvspace{1cm}

		Sollten von Ihnen Abbuchungsaufträge für Mitgliedsbeitrags- und Pachtzahlungen erteilt
		worden sein, bitten wir diese entsprechend zu berücksichtigen. 
		
		\closing{Mit freundlichen Grüßen}
		
		%\ps PS: HDL % wenn gesetzt, dann PS- Zeile; S. 166

	\end{letter}

\end{document}